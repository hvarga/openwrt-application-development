\section{Modules Overview}

\subsection{List of Core Modules}
\begin{frame}{List of Core Modules}
    \pause
    \begin{description}[<+-|alert@+>]
        \item[ubox] General purpose library which provides features like an event loop, binary blob message formatting and handling, linked list, and some JSON helpers~\cite{openwrt-ubox}~\cite{openwrt-libubox}~\cite{lutfi-ubox_ubus}~\cite{openwrt-log}.
        \item[uci] Configuration interface aimed to centralize the configuration of OpenWrt~\cite{openwrt-uci}~\cite{openwrt-libuci}.
        \item[ubus] Sofware communication bus which provides communication between various daemons and applications in OpenWrt~\cite{openwrt-ubus}~\cite{lutfi-ubus}~\cite{lutfi-ubox_ubus}~\cite{openwrt-rpc_guide}~\cite{openwrt-rpc_techref}.
        \item[procd] Process manager~\cite{openwrt-procd}.
        \item[opkg] Package manager~\cite{openwrt-opkg}.
    \end{description}
\end{frame}

\subsection{Logical Structure}
\begin{frame}{Logical Structure}
    \fitframegraphics{openwrt-modules}
\end{frame}

\section{ubox}

\subsection{Overview}
\begin{frame}{Overview}
    \begin{itemize}[<+-|alert@+>]
        \item Collection of C utility functions for OpenWrt.
        \item Composed of two components: \href{https://git.openwrt.org/project/libubox.git}{libubox} and \href{https://git.openwrt.org/project/ubox.git}{ubox}.
        \item General purpose library (think \href{https://wiki.gnome.org/Projects/GLib}{glib} and \href{https://www.boost.org/}{boost} only micro size) and utilities.
        \item Provides polling, event handling, socket helper functions, logging, MD5 hashing, generating/parsing tagged binary data, AVL binary-tree balancing, JSON handling, task queueing/completion tracking, endian conversion, Base64 encoding/decoding, linked lists and key value lists.
        \item Provides centralized logging systems with logd (think Syslog) and logread.
    \end{itemize}
\end{frame}

\subsection{uloop}

\subsubsection{Problem Domain}
\begin{frame}{Problem Domain}
    \fitframegraphics{event-loop}
\end{frame}

\subsubsection{Features}
\begin{frame}{Features}
    \begin{itemize}[<+-|alert@+>]
        \item I/O event notification loop.
        \item Its job is to poll file descriptors, run timers, manage child processes.
        \item Event loop is backed by epoll and kqueue~\cite{starch-linux-programming-interface}.
    \end{itemize}
\end{frame}

\subsubsection{Code Snippet}
\begin{frame}[fragile, allowframebreaks]{Code Snippet}
    \lstinputlisting[language=C]{uloop-hello-world.c}
\end{frame}

\section{uci}

\subsection{Overview}
\begin{frame}{Overview}
    \begin{itemize}[<+-|alert@+>]
        \item Common configuration format used by every application running on OpenWrt.
        \item Centralize the whole configuration of a device running OpenWrt.
        \item Configuration file is exposed with the \inlinecode{uci} utility and \inlinecode{libuci} C library.
        \item New application should use the uci system.
        \item Existing non-uci compatible application can be made compatible by writing wrapper.
        \item Eases application development since developer doesn't need to bother with creating his own configuration syntax and parser.
        \item Since every other application shares the common configuration format, less time is spent to accomodate to a yet another configuration syntax.
    \end{itemize}
\end{frame}

\subsection{Features}
\begin{frame}{Features}
    \begin{itemize}[<+-|alert@+>]
        \item OpenWrt central configuration is split into files located in the \inlinecode{/etc/config/} directory.
        \item Each configuration file relates roughly to the part of the system it configures or the application.
        \item Configuration file can be modified using text editor or \inlinecode{uci} command line tool. Custom C application that uses \inlinecode{libuci} can be used for modification.
        \item \inlinecode{uci} tool exposes CRUD operations on configuration file.
        \item \inlinecode{uci} tool CRUD operation are staged into a temporary location and written permanently to flash with a commit command.
    \end{itemize}
\end{frame}

\subsection{Configuration File Syntax}
\begin{frame}{Configuration File Syntax}
    TODO
\end{frame}
